\documentclass[a4paper,12pt]{ltjsarticle}
\usepackage{graphicx}
\usepackage{amsmath, amssymb}
\usepackage{geometry}
\usepackage{enumitem}
\usepackage{luatexja-fontspec}
\usepackage{hyperref}
\usepackage{color}
\setlist[itemize]{label=-}
\setmainjfont{IPAexMincho}
\setmainfont{Times New Roman}
\setsansjfont{IPAexMincho}
\setsansfont{Times New Roman}
\geometry{margin=20mm}
\renewcommand{\baselinestretch}{1.0}
\setlength{\parskip}{0.5em}
\setlength{\parindent}{1em}

\begin{document}
% \section*{4つのポイント}
% \begin{itemize}
%     \item コミュニケーション
%     \item 言わせてホメる
%     \item 短く・早く
%     \item 確認してホメる
% \end{itemize}

\section*{カラーについて}
2026\_1学期に残る先生(暫定)
\begin{itemize}
    \item うっちー
    \item わっくん
    \item ぬんちゃん(休職)
    \item ハーカ
    \item つっちー
    \item ハマ(休職)
    \item やーまん
    \item じぇん
    \item アニ
    \item ばっさー
    \item ムッタ
    \item いがちゃん
    \item てんしん
    \item けんけん
    \item えいちゃん
    \item やまぴー
    \item (こころ)
    \item (しえな)
    \item (なおと)
\end{itemize}

上記で19人(17人)\\
↓\\
30%は6人(休職は除くと5人)\\

\newpage
\subsection*{カラーはどうなる?}
うっちーの個人的な考え\\

既存のカラー
\begin{itemize}
    \item うっちー(Y)
    \item じぇん(2026\_1にBにしたかったけど無理?)
    \begin{itemize}
        \item メンター業務やってない?つまりは2026\_2にBでOK?\\
    \end{itemize}
\end{itemize}

新規カラー
\begin{itemize}
    \item つっちー
    \begin{itemize}
        \item 金(うっちー)
        \begin{itemize}
            \item フロントコミ
            \item 成績(4つのポイント,授業時間)
            \item マイスター(取る前にカラーでOKだけど,取る方向で)\\
        \end{itemize}
    \end{itemize}
    \item ばっさー(本人と明日話す)
    \begin{itemize}
        \item 木(ぬんちゃん),土(うっちー)
        \begin{itemize}
            \item フロントコミ
            \item 先生コミ
            \item 授業は今のままでGOODなので,マイスター取らせましょう\\
        \end{itemize}
    \end{itemize}
    \newpage
    \item けんけん(どうなってる?)
    \begin{itemize}
        \item 月(ぬんちゃん),木(うっちー?)
        \begin{itemize}
            \item フロントコミ
            \item 先生コミ
            \item 授業は今のままでGOODなので,マイスター取らせましょう\\
        \end{itemize}
    \end{itemize}
    \item ヤーマン?
    \begin{itemize}
        \item 火(ぬんちゃん),水(じぇんサポ?)
        \begin{itemize}
            \item フロントコミ
            \item 成績?\\
        \end{itemize}
    \end{itemize}
    \item ムッタ
    \begin{itemize}
        \item 金(うっちー)
        \begin{itemize}
            \item フロントコミ(今でいい感じ?)
            \item 成績(4つのポイント)
            \item マイスター(上期取ってから本格始動?)\\
        \end{itemize}
    \end{itemize}
\end{itemize}

\newpage
つっちーは,学年的な問題でカラーを1人置いておきたい気持ちがある+本人のやる気あるから\\

ばっさーは,「こころ」と「しえな」がやるとしたら女性の先輩がいた方が校舎として良い(昨日話した)\\

上記の2名は2026\_1学期からカラーを目指してくれると大変ありがたい\\

----------\\

けんけんは,わかんないけどやる気を示してくれてるならサポートするべきでは?って思う\\

やーまんは,まだ本格的には話してない?どんな状態?\\

ムッタは,本人のやる気と頑張りたい気持ちあり.基本OPから見直して,成績面で勝てるようになろう\\\\

基本的には,うっちーorぬんちゃんで3月まで見る(今後見据えて).この辺の話を早急にカラーで共有しないと,せっかく強いカラーがいるのに継承できずに終わりますよ?

\newpage
今後のメモ(何やる?)
\section*{イエロー業務リスト}
\begin{itemize}
    \item 電話対応
    \item 夕礼
    \item サポートシート(新人tr)
    \item 宿題忘れ対応
    \item テスト後面談
    \item 高校生面談
    \item 提案書面談(本人OK)
    \item 模試後タッチ
\end{itemize}

\end{document}
