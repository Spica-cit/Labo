\documentclass[a4paper,12pt]{ltjsarticle}
\usepackage{graphicx}
\usepackage{amsmath, amssymb}
\usepackage{geometry}
\usepackage{enumitem}
\usepackage{luatexja-fontspec}
\usepackage{color}
\usepackage{hyperref}
\setlist[itemize]{label=-}
\setmainjfont{IPAexMincho}
\setmainfont{Times New Roman}
\setsansjfont{IPAexMincho}
\setsansfont{Times New Roman}
\geometry{margin=15mm}
\renewcommand{\baselinestretch}{1.2}
\setlength{\parskip}{0.5em}
\setlength{\parindent}{1em}

\begin{document}
\centerline{\Large 古池先生mtg資料}
\rightline{\mid \today}
\rightline{\mid 千葉工業大学大学院}
\rightline{\mid 学習工学研究室}
\rightline{\mid 内山裕太}

\section*{今回の報告内容}
\begin{itemize}
    \item 勉強の進捗
    \item 日毎のスケジュール見直し
\end{itemize}

\section*{勉強の進捗}
現在読んでいる論文\\
平嶋宗: 学習課題の内容分析とそれに基づく学習支援システムの設計・開発:算数を事例として, 教育システム情報学会誌, Vol. 30, No. 1, pp. 8-19, 2013\\

先日の白髭先輩への報告\\
----------\\

今日勉強した内容で非常に重要な内容があって,「学習科学」と「学習支援システムの設計」は観点が異なるってことです.「紙とペン,黒板とチョークでは実現が難しいものを実現する必要がある」という言葉に大きな関心を受けました.この「支援システム開発において当たり前のこと」が自分の中に抜けていたかもしれないです.\\

逆に言えば,学習活動自体の論理的な説明は中途半端で良くて,大切なのは「従来の環境での学習で難しかったことの提示」と「具体的にどのように改善できるのかの考えとその効果」が示せることです.\\

まだ最初の3ページほどしか読み込めてませんが,Carbonellらの論文は読んだ方がいいかもしれないと思いました.「情報構造指向」という言葉が理解しきれなかったので...(ぱっと読んだ感じではめっちゃ大切そうでした)\\

同時に,Papertの「MIND-STORMS」という文献も大切そうです.訳されたものがあるらしい?ので調べてみますが,「知識を頭に入る大きさにする」ことこそが「知識の伝達,組み立て」において重要らしいです.自分の研究では「知識の組み立て」は非常に重要な観点であり,新たな知識をどの程度まで砕くことで獲得させられるかは知っておく必要があると感じました.\\

pdfで読むより,紙で読んでノートにまとめながら読むと「眠くならない」かつ「思ったことをすぐに外化できる」という点で非常に良いと気付きました.\\

とりあえず,平嶋先生から「課題分析が足らんな」と言われて終わりそう(想像)なので,もう少し自分の領域における課題分析をしっかりやってみる必要がありそうです.というか,課題分析がめっちゃしっかりしてるからこそ,平嶋先生の論文は理解するのが難しいんだと感じました.論文を読むときはその著者が解決しようとしている課題が何なのかを読み取る必要がありそうです.逆に,原稿などを書くときは,拝読者にそれらが伝わるように書かないと読まれずに終わりそうです.\\

----------

上記に加えて,自身が勉強で書いたノートと論文の該当箇所のpdfデータを添付しておきます(今回の「こんな感じで勉強してます」という意味合いで添付します)

今後読む予定の文献は以下の通りで考えています

\begin{itemize}
    \item 東本崇仁: 私の教育システム情報学マップ…を作るに至るまでの話, 教育システム情報学会誌, Vol. 39, No. 2, pp. 179-186, (2022)
    \item 瀬田和久, 桑原千幸, 仲林清: 採録される論文の書き方-誌上チュートリアル-, 教育システム情報学会誌, Vol. 38, No. 2, pp. 82-93, (2021)
    \item Carbonell, J. R.: AI in CAI: an artificial intellgence approach to computer-assisted instruction, IEEE Transaction on Man-Machine Systems, Vol. 11, No. 4, pp. 190-202, (1970)
    \item 平嶋宗, 中村祐一, 池田満, 溝口理一郎, 豊田順一: ITSを指向した問題解決モデルMIPS, 人工知能学会誌, Vol. 7, No. 3, pp. 475-486, (1992)
\end{itemize}

Carbonellの論文は,内山の英語能力がゴミすぎるせいで,拝読に相当時間がかかるかもしれません(重要ではあるが,後回し?)
\clearpage

\section*{日毎のスケジュール見直し}
先日東本先生とSlack上でやり取りを行い,「メンター活動にかける時間を決める」という目標を自分の中で決めました.具体的には平日の13:00〜14:00までは後輩の指導を優先的に行う時間にしようと思っています.実験関係などの急を要するものは適宜対応しますが,基本的にはこの方針でやろうと思います.\\

また,朝方の人間になろうキャンペーンを始めました.早く起きて,早く研究室に来るようにがんばります(今日は8:15に研究室につきましたが,生活スケジュールがズレたせいか1日気持ち悪かった).残りの1年は早くきて活動する方向で頑張った方が,卒業後のこと考えても良さそうなので,一旦年末まで頑張ってみようと思います.守れないことがあるかもしれませんが,今が変わりどきかなと思います.

\end{document}
