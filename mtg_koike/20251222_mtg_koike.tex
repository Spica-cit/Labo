\documentclass[a4paper,12pt]{ltjsarticle}
\usepackage{graphicx}
\usepackage{amsmath, amssymb}
\usepackage{geometry}
\usepackage{enumitem}
\usepackage{luatexja-fontspec}
\usepackage{color}
\usepackage{hyperref}
\setlist[itemize]{label=-}
\setmainjfont{IPAexMincho}
\setmainfont{Times New Roman}
\setsansjfont{IPAexMincho}
\setsansfont{Times New Roman}
\geometry{margin=15mm}
\renewcommand{\baselinestretch}{1.2}
\setlength{\parskip}{0.5em}
\setlength{\parindent}{1em}

\begin{document}
\centerline{\Large 古池先生mtg資料}
\rightline{\mid \today}
\rightline{\mid 千葉工業大学大学院}
\rightline{\mid 学習工学研究室}
\rightline{\mid 内山裕太}

\section*{今回の報告内容}
\begin{itemize}
    \item 就活関係
    \item 前回共有していただいた論文の2章を拝読
\end{itemize}

\section*{就活関係}
今週の前半は就活関係に時間を使っていました.

\section*{前回共有していただいた論文の2章を拝読}
Carbonellの話の際にいただいた論文の2章を読みました.\\

ドメインモデル:ITSの対象領域の知識モデル

学習者モデル:学習者の理解状態を表現し,学習状態を捉えるためのモデル

教授モデル:上記2つのモデルを合わせ,何を教えるかを決定するモデル\\

これらは,設計指針の1つであり,ドメインモデルは特にITSの根幹を担う部分\\

認知スキーマとは?(認知スキーマ理論:CST)\\

学習者が持つ知識や思考の枠組みのこと

→新しい情報などを解釈するための基盤となる\\

思考とは情報の構造(認知スキーマ)に対する操作(Derryの論文目を通した方が良い?)\\

認知スキーマは理解した方が良い(to 内山)\\\\

学習者って「何を考えて」「何を根拠に」解答を作るんだろう(あまり気にしてなかった)

\section*{メモ}
朝起きようキャンペーンは継続中

朝に起きれるようになってきたので,そこからいかに早く研究室に到着し,作業を始められるかの勝負になりそうです.年末休みに入ったら,1日だけゆっくり寝たいと思ってます.\\

最近感じたのが,数学を題材として「抽象化」「具体化」という内容を扱うのであれば,方程式とかを題材にした方が良い?深く考えられてないです.

\end{document}
