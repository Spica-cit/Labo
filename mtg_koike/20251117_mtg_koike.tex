\documentclass[a4paper,12pt]{ltjsarticle}
\usepackage{graphicx}
\usepackage{amsmath, amssymb}
\usepackage{geometry}
\usepackage{enumitem}
\usepackage{luatexja-fontspec}
\usepackage{hyperref}
\usepackage{color}
\setlist[itemize]{label=-}
\setmainjfont{IPAexMincho}
\setmainfont{Times New Roman}
\setsansjfont{IPAexMincho}
\setsansfont{Times New Roman}
\geometry{margin=15mm}
\renewcommand{\baselinestretch}{1.2}
\setlength{\parskip}{0.5em}
\setlength{\parindent}{1em}

\begin{document}
\centerline{\Large 古池先生mtg資料}
\rightline{\mid \today}
\rightline{\mid 千葉工業大学大学院}
\rightline{\mid 学習工学研究室}
\rightline{\mid 内山裕太}

\section*{本日の報告内容}
\begin{itemize}
    \item ALSTを終えて
    \item 最近の活動状況
\end{itemize}

\section*{ALSTを終えて}
古池先生はお気付きだったかもしれませんが,ALSTのポスター発表で山本先生との質疑応答を行なってボコされました.辛かったです.

自分の無力さを痛感して,色々勉強しないとなって思いました.そこで,以下の文献を一通り読む時間に入っています.

\begin{itemize}
  \item 勉強する(拝読中:\color{blue}青\color{black},拝読済:\color{red}赤\color{black})
    \begin{itemize}
        \item 古池研究室の研究の進め方
        \item 認知学習工学入門
        \item 数学問題における論理構造の可視化機能を用いた学習支援システムの開発
        \item 逆思考型を対象とした算数文章題の作問学習支援システムの設計開発と実践的利用
        \item ITSを指向した問題解決モデルMIPS
        \item 補助問題の定式化
        \item 部品からの構造再編成を通した学習-マルチモーダル統合と部品・構造・体験の共有化-
        \item \color{blue}理科系の作文技術\color{black}
    \begin{itemize}
      \item これの外化サイト作ってみようかな(面白そう)
    \end{itemize}
  \end{itemize}
\end{itemize}

今までに読んだことがある論文を含めて,一度読み直すとともに,自身の研究を振り返ってみる時間が必要だと考えました.

また,以下の資格を卒業までに取得できればと考えました.取れるとは言い切れません.

\begin{itemize}
    \item (ITパスポート)
    \item 基本情報技術者試験
    \item 応用情報技術者試験
    \item マイクロソフトオフィススペシャリスト(簡単そう)
    \item ファイナンシャルプランナー2級?(楽しそう)
    \item TOEIC
    \item etc...
\end{itemize}

\section*{最近の活動状況}
最近は研究に対する力が入らず,今後のスケジュールや内山の私的なシステム開発に時間を使ってしまっています(ALSTで色々大変だったので,一時的に自分のやりたいことをやる時間も良いのかなと思って動いています).

また,スケジューリングがうまくいかないことや,ALSTなどによるメンタル状況を加味して,前田先輩と簡易的なmtgを行なっていただきました.心配をおかけしてしまいました...

11/25(火)の夜に就職先の偉い人(実力者)と今後の動き方と理想についてお話しする時間をいただきました(2時間くらい)

頑張って卒業できるように計画立てします...

\end{document}
