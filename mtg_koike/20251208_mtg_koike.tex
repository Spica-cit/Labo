\documentclass[a4paper,12pt]{ltjsarticle}
\usepackage{graphicx}
\usepackage{amsmath, amssymb}
\usepackage{geometry}
\usepackage{enumitem}
\usepackage{luatexja-fontspec}
\usepackage{color}
\usepackage{hyperref}
\setlist[itemize]{label=-}
\setmainjfont{IPAexMincho}
\setmainfont{Times New Roman}
\setsansjfont{IPAexMincho}
\setsansfont{Times New Roman}
\geometry{margin=15mm}
\renewcommand{\baselinestretch}{1.2}
\setlength{\parskip}{0.5em}
\setlength{\parindent}{1em}

\begin{document}
\centerline{\Large 古池先生mtg資料}
\rightline{\mid \today}
\rightline{\mid 千葉工業大学大学院}
\rightline{\mid 学習工学研究室}
\rightline{\mid 内山裕太}

\section*{今回の報告内容}
\begin{itemize}
    \item 研究調査
    \item 自身の状態について
\end{itemize}

\section*{研究調査}
数学分野が嫌いになったわけではないですが,関数分野を対象としているから嫌になっているのかなと思ったりしています.

自分が塾講師をしていて感じた問題点について考え,研究テーマになりそうだと思ったものをまとめてみました.

\color{blue}近年は推薦入試での受験が増加しているため,小論文や作文の利用が増えている.しかし,学校のカリキュラムとしてそれらの学習を独立的に教えている場は少ない.\color{black}
\clearpage

\section*{研究計画書}
\section{研究名(仮)}
小論文などの文章表現向上を目的とした学習支援システムの設計と開発

\section{研究背景}
近年,大学や高校の入学試験において,小論文などの利用が増加している傾向にある.しかし,学校における授業カリキュラムの中でそれらをメインに扱うことが少なく,希望者のみが指導を受ける形になることが多い.その結果,学習者個人の文章表現スキルに応じた支援が十分に行えていない可能性がある.

文章作成能力自体は,大学での卒業論文作成等でも必要とされる力である.また,自身の考えを的確に文章として表現する力を獲得することで,他者が書いた文章の主張と根拠を読み取る力が育まれる可能性が考えられる(推測であり憶測).自身の考えを文章として外化するためには,自身の考えを論理的に構築した上で,考えを文章として表現するための力が必要となる.ここで,現状考える問題点について4点示す.

\begin{itemize}
    \item 考えを的確に文章化できない
    \item 表現的な問題により意図した伝達ができない
    \item 添削作業が属人的(個体差がある)
    \item 学習者が添削内容を認識しづらい
\end{itemize}

\section{研究目的}
小論文や作文などの文章作成において,学習者が文章化できない原因(構造や表現,論理)に対し,的確なFBを提示する学習支援システムを設計し,効果を測ること

\begin{itemize}
    \item 文章の自動解析(論理や表現)
    \item FB設計
    \item 効果検証
\end{itemize}

\section{スケジュール}
\begin{itemize}
    \item 研究調査(1月まで)
    \item システム,FB設計(3月まで)
    \item システムプロトタイプ開発(4月まで)
    \item 実験準備(5月前半)
    \item 実験プレ(6月まで)
    \item 実験実施(8月まで)
    \item 分析(9月まで)
    \item 追加実験考察と準備(10月まで)
    \item 実験実施(11月まで)
    \item 分析(12月前半)
    \item 修論執筆(12月前半から1月前半)
\end{itemize}
\clearpage

\section*{自身の状態について}
最近ストレスを感じることが多く,研究が思うように進んでいませんでした

少しはやらないといけないといけないと思って,今回の小論文の話を持ってきました.数学の別の分野を調べてみるのもありだと思ってます(十文字先輩のベクトルとか,黒川さんのグラフ系とか)

\section*{余談}
システムを作ってみました

\color{blue} 模試の結果と志望校情報から復習単元を予測するサイト

\href{https://spica-cit.github.io/Results_analysis/}{https://spica-cit.github.io/Results\_analysis/}

\end{document}
