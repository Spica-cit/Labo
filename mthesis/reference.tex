% \chapter*{受賞・学域活動一覧}
% \addcontentsline{toc}{chapter}{受賞・学域活動一覧}

% 受賞
% \begin{description}
%    \item[2024.9] 人工知能学会 第6回SIAI 産学クロススクエア「ミライをつくるAI人材」大学研究室展示アワード, (2024/09) (受賞者:相川野々香,茂木誠拓,前田新太郎,中村祐希人,白髭虹輝,内山裕太,PIERRET ROBLES JEAN PAUL)
% \end{description}

\chapter*{業績一覧}
\addcontentsline{toc}{chapter}{業績一覧}

業績の概算(改稿中,発表予定など含む)
\begin{description}
   % \item[学術論文(査読有)] 0報
   % \item[国際会議(査読有)] 0報
   \item[研究会] -報
   \item[全国大会] -報
   \item[その他] -報
   \item[合計] --- -報
\end{description}

% 学術論文(査読有)
% \begin{enumerate}
%    \item なし(消していい?)
% \end{enumerate}

% 国際会議(査読有)
% \begin{enumerate}
%    \item なし(消していい?)
% \end{enumerate}

研究会
\begin{enumerate}
   \item 内山裕太, 白髭虹輝, 古池謙人, 東本崇仁: 数学学習における解法の構造理解と再利用を可能とする学習手法の提案と学習支援システムの開発と評価, 教育システム情報学会 (JSiSE) 2024年度第6回研究会, Vol. 39, No. 6, pp. 91-98, (2025/03)
   \item 内山裕太, 白髭虹輝, 古池謙人, 東本崇仁: 部品の段階的拡張手法を活用した中学数学の関数分野における解法の構造理解と再利用を促す学習支援システムの設計・開発, 人工知能学会第103回先進的学習科学と工学研究会, pp. 15-20, (2025/03)
   \item 内山裕太,古池謙人,東本崇仁: 問題構造に基づく構造の部分的再利用を指向した学習支援システムの設計・提案, 人工知能学会大105回先進的学習科学と工学研究会, pp. 42-46, (2025/11)
\end{enumerate}

全国大会
\begin{enumerate}
   \item 内山裕太,古池謙人,東本崇仁: 中学数学の関数分野における解法の構造理解と部品的知識の獲得・再利用を指向した学習支援システムの提案, 教育システム情報学会全国大会講演論文集, pp. 97-98, (2024/08)
\end{enumerate}

その他(予稿無し発表など)
\begin{enumerate}
   \item 内山裕太: 中学数学における関数を題材とした学習支援システムの開発・評価, 教育システム情報学会関東支部企画 第5回リサーチ・コ・コ・コモンズ, (2023/11)
   \item 内山裕太: 中学数学の関数分野における解法の構造理解と部品的知識の獲得・再利用を指向した学習支援システムの開発, 教育システム情報学会関東支部企画 第5回リサーチ・コ・コ・コモンズ, (2024/12)
\end{enumerate}
