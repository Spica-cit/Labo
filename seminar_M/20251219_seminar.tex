\documentclass[a4paper,12pt]{ltjsarticle}
\usepackage{graphicx}
\usepackage{amsmath, amssymb}
\usepackage{geometry}
\usepackage{enumitem}
\setlist[itemize]{label=-}
\usepackage{luatexja-fontspec}
\setmainjfont{IPAexMincho}
\setmainfont{Times New Roman}
\setsansjfont{IPAexMincho}
\setsansfont{Times New Roman}
\geometry{margin=25mm}
\usepackage{hyperref}
\usepackage{color}

\begin{document}

\centerline{\huge 進捗報告 ver.33}

\rightline{\large \today}
\rightline{\large 千葉工業大学大学院}
\rightline{\large 学習工学研究室}
\rightline{\large 内山裕太}

\section*{今回の進捗}
\begin{itemize}
  \item 講義課題(メディア情報処理特論)
  \item 卒論チェック
  \item 就職関係
\end{itemize}

\subsection*{講義課題}
時間かけすぎない程度で終わらせました.FB返ってくるの待って,合格基準超えてたら修正なしで終えます.

\subsection*{卒論チェック}
高野,紫関,仲地の卒論をチェックしました.勝井については,実験関係を最優先で動いてもらったため,ほぼ見れていませんが,実験が始められるので空き時間に書いてもらいます.\\

以前,前田先輩に言われた通り,「時間をかけすぎない」を前提にやり切りました(20〜30分ずつくらい).多かったのが,「主述関係がずれている」「結局何が言いたいの?」「はじめにで主張している内容と提案システムの内容って合ってる?」などです.他のメンターに比べてしっかり見れている自信はないですが,去年の自分よりは見れていると思います.\\

学会原稿執筆が始まってくるので,手を抜かずに自分の全力を出したいです.

\subsection*{就職関係}
来年から研修等が始まるのですが,少しずつ目標を決めて動くようになってます.興味ないと思うので詳細は省きますが,研究室活動との両立が結構大変だなと思えてきました.不思議なのが,「研究室に来る時間がない」といった「物理的な難しさ」ではなく,「考えることが多すぎる」といった難しさなんですよね(伝われ...).なんか,日頃から頭使って色々考えてはいるんだなって思いました.\\

たまに「早く働きたいな」って思うことがあって,「院に行かなければもう働けてたのに」とか考えることがあります.ただ,色々なことに問題を感じたり,自分なりに解決策を考えたりといった部分で,「院に行ったからこそ得られた問題解決能力もあるよな」って結論に至ります.つまりは,院に進んだことによるメリットの方が大きかった(現時点で)ということだと思います.東本先生ありがとうございます.

\section*{今後の計画}
\begin{itemize}
  \item 研究勉強
  \item 研究方針の制定
\end{itemize}

\subsection*{研究勉強}
古池先生と話したところ,Carbonellは言うほど全てを読み込むほどの時間を注ぎ込むものではないと言われました.古池先生が関わっている研究会原稿の一部をお勧めされたので,その部分を読んでみます.また,勉強をするのに朝がとても有効であることがわかりました.卒論チェックなども朝にやったところ,めちゃめちゃ捗ったのでおすすめします.

\subsection*{研究方針の制定}
今まで関数学習を題材として「抽象化って大事だよ」って主張してきたんですが,いまいちピンとこないところがあったんですよね(理系だから仕方ないのかな...).最近感じたのが,方程式とかを問題文から作って解く問題とかで,条件とか問題背景情報が変わると解けない学習者って意外と多い.この辺とかまさに抽象化することが大事で,例えば「道のりは,速さと時間をかける」といった考え方はその問題だから成り立つのであって,本来は「この変数を求めるためには,問題文にある変数と変数をかける」という風に考えた方が良い?この辺とかをしっかり議論できたらそれっぽいのできそう(修士研究は「数学における抽象的思考の重要性」とかで結果出して終わりそう.残りはMr.藤島に任せよう).\\

年末年始で,冬季講習中に現役中高生に「数学って何が難しいの?」って聞いてみます.聞き方間違えるとウザイやつになりそうなので,気をつけます.

\section*{問題点・課題}
\begin{itemize}
  \item 3月の学会は発表者として参加はしないが,後輩の勇姿を見に行くか検討中
\end{itemize}

\section*{雑談}
高校の推薦入試狙いって結構大切ですよね.

\section*{メモ}

\end{document}
