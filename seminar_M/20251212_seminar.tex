\documentclass[a4paper,12pt]{ltjsarticle}
\usepackage{graphicx}
\usepackage{amsmath, amssymb}
\usepackage{geometry}
\usepackage{enumitem}
\setlist[itemize]{label=-}
\usepackage{luatexja-fontspec}
\setmainjfont{IPAexMincho}
\setmainfont{Times New Roman}
\setsansjfont{IPAexMincho}
\setsansfont{Times New Roman}
\geometry{margin=25mm}
\usepackage{hyperref}
\usepackage{color}

\begin{document}

\centerline{\huge 進捗報告 ver.32}

\rightline{\large \today}
\rightline{\large 千葉工業大学大学院}
\rightline{\large 学習工学研究室}
\rightline{\large 内山裕太}

\section*{今回の進捗}
\begin{itemize}
  \item 活動スケジュールの設定
  \item 会議室連絡の作成
  \item 勉強
  \begin{itemize}
    \item 平嶋宗: 学習課題の内容分析とそれに基づく学習支援システムの設計・開発:算数を事例として, 教育システム情報学会誌, Vol. 30, No. 1, pp. 8-19, 2013
  \end{itemize}
\end{itemize}
\clearpage

\subsection*{活動スケジュールの設定}
自分の活動を犠牲にしてメンター活動とかしてしまう自分の活動の仕方を見直しました.

急を要する相談の場合は別途時間をとりますが,基本的には13:00〜14:00をメンター活動時間として設定しました(試験的).活動しづらかったりしたら変更します.

(朝早く起きるようにしてください to 内山)

\subsection*{会議室連絡の作成}
現段階では内山のGoogleアカウント上でGASを構築している.Slackのアプリ(botアカウント)も内山のアカウントで作成.今後継承するなら研究室のアカウントとかでやらないといけないかも(内山が今後消さなければ大丈夫ですが,いつ内山が学習工学研究室から消えるかわからないので)

機能面で気になったところを都度修正してますが,一旦良さそうな形に落ち着いています.

\subsection*{勉強}
\#白髭メンターからコピペ

----------\\

今日勉強した内容で非常に重要な内容があって,「学習科学」と「学習支援システムの設計」は観点が異なるってことです.「紙とペン,黒板とチョークでは実現が難しいものを実現する必要がある」という言葉に大きな関心を受けました.この「支援システム開発において当たり前のこと」が自分の中に抜けていたかもしれないです.\\

逆に言えば,学習活動自体の論理的な説明は中途半端で良くて,大切なのは「従来の環境での学習で難しかったことの提示」と「具体的にどのように改善できるのかの考えとその効果」が示せることです.\\

まだ最初の3ページほどしか読み込めてませんが,Carbonellらの論文は読んだ方がいいかもしれないと思いました.「情報構造指向」という言葉が理解しきれなかったので...(ぱっと読んだ感じではめっちゃ大切そうでした)\\

同時に,Papertの「MIND-STORMS」という文献も大切そうです.訳されたものがあるらしい?ので調べてみますが,「知識を頭に入る大きさにする」ことこそが「知識の伝達,組み立て」において重要らしいです.自分の研究では「知識の組み立て」は非常に重要な観点であり,新たな知識をどの程度まで砕くことで獲得させられるかは知っておく必要があると感じました.\\

pdfで読むより,紙で読んでノートにまとめながら読むと「眠くならない」かつ「思ったことをすぐに外化できる」という点で非常に良いと気付きました.\\

とりあえず,平嶋先生から「課題分析が足らんな」と言われて終わりそう(想像)なので,もう少し自分の領域における課題分析をしっかりやってみる必要がありそうです.というか,課題分析がめっちゃしっかりしてるからこそ,平嶋先生の論文は理解するのが難しいんだと感じました.論文を読むときはその著者が解決しようとしている課題が何なのかを読み取る必要がありそうです.逆に,原稿などを書くときは,拝読者にそれらが伝わるように書かないと読まれずに終わりそうです.\\

----------
\clearpage

\section*{今後の計画}
\begin{itemize}
  \item 勉強を進める(読みたい論文は選定済み,とりあえずCarbonell)
\end{itemize}

\section*{問題点・課題}
\begin{itemize}
  \item 早く起きろよ?やる気あんのか?内山
  \item 勝井君が心配
\end{itemize}

\section*{雑談}
塾講師として日本1位の成績を出すと1学期間金の名札になるらしい(逆に嫌だ)

\section*{メモ}
方程式とか規則性の問題とかは構造的な類似に気づかせやすいかも?関数は面倒(基本的に似ているけど,構造的にではなく数字が違うだけだから)\\

何気ない会話の中で「こうやって教えたらおもしろくね?わかりやすくね?」って思うことをシステム化して欲しい(東本先生)\\

↑塾の担当生徒に聞いてみます(実際にわからない人に聞いてみた方が面白そう)\\

エクセルの基本的な使い方がわからない人が多いらしい(は?)

東本先生と話しましたが,情報系卒として終わってるレベルだと思います(言い方悪いが,文系でもできる...)\\

自分的には「検定の0.05って何を意味するの?」とか「結果が出なかったけど,何が原因?」とか色々気にしちゃうので,すぐに諦めてしまうのは勿体無いと思うし,探究心ないなぁって思います.\\\\

学会は...silence...\\\\

昔のシステムは概念がなかった(ただのデータ).機械学習のすごいところは動的に生成することができること(ノードやエッジを人間が作っているから,人間のわかる言葉で説明される)だが,ロジカルに妥当な説明が難しい.

\end{document}