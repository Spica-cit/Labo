\documentclass[a4paper,12pt]{ltjsarticle}
\usepackage{graphicx}
\usepackage{amsmath, amssymb}
\usepackage{geometry}
\usepackage{enumitem}
\setlist[itemize]{label=-}
\usepackage{luatexja-fontspec}
\setmainjfont{IPAexMincho}
\setmainfont{Times New Roman}
\setsansjfont{IPAexMincho}
\setsansfont{Times New Roman}
\geometry{margin=25mm}
\usepackage{hyperref}
\usepackage{color}

\begin{document}

\centerline{\huge 進捗報告 ver.34}

\rightline{\large \today}
\rightline{\large 千葉工業大学大学院}
\rightline{\large 学習工学研究室}
\rightline{\large 内山裕太}

\section*{---報告内容---}
年末年始にかけて,様々な問題が起こったことにより,院ゼミやディスカッション,卒業審査関係の研究室行事にご迷惑をおかけしました.体調不良以外の事の詳細については,少しずつ片付けるつもりです.\\

卒業までのスケジュールを立てようと思いましたが,研究の方向性考えたら少しぶれそうだったので,3月までの流れを考えてみました.
\clearpage

\large 1月
\begin{itemize}
  \item 研究の方向を固める(前に考えた「方程式を用いた抽象化」の話)
  \item 研究内容を加味した計画書を作る(文献調査~実験まで)
  \item システム案を立てる(おそらく一番時間かかる)
\end{itemize}

\large 2月
\begin{itemize}
  \item システム機能要件表を作る
  \item 開発環境を決める(多分C\#)
  \item システム開発開始
\end{itemize}
メモ\\
2/17 研修の可能性あり(高)\\

\large 3月
\begin{itemize}
  \item 半ばくらいに雑なシステムを完成させたい
  \item 手法とシステム状態に対して議論(あくまでも理想論)
\end{itemize}
メモ\\
3/3 研修?で不在の可能性(高)\\

一旦,3月の学会ではなく夏の全国大会で実験の結果について触れたいなくらいで思ってます.予備実験と呼ぶにも,何を調べる予備実験なのか考えられていないので,とりあえずシステム案作って提出してみます.自分勝手な発言ではありますが,3月の学会に登録すると,今の自分の状況的にまた病みそうな気がしてきてて怖いです.

\section*{---今後の動き---}
\begin{itemize}
  \item 方程式方面の研究を不可ぼってみる(抽象化の主張)
  \item モンサクンちゃんと読んだ方がよさそう
  \item モンサクンの参考文献も調べる
\end{itemize}

\section*{---問題点・課題---}
\begin{itemize}
  \item 院ゼミが火曜日に変わったと勘違いしており,金曜日だけ16:25退出になってしまった(2月末まで)
  \item ↑の代わりに月水がフリーになりました(テスト前以外)
\end{itemize}

\section*{---雑談---}
僕はあなたを許さない

\section*{---メモ---}
今年色々指導していて,方程式を文章から立てて解く問題の方が抽象化の考え方大切じゃね?ってなりました.

学校から動物園に行く問題で,帰り道に公園を経由するとわからなくなるのは何ででしょうか?基本的には問題に含まれる要素(オブジェクトなのか?)を抜き取って,関係性を構築して,計算するって方向は変わらないはずなんですが...

↓

となると,白髭先輩のやってることって,僕が数学で重要だと思っていることにとても近い!?ってなってます.やはり,数学にはプログラミング的思考が関連してるんだと思います.

昨日,白髭先輩の実験を受けて,数学でも似たような力必要じゃね?って思って受けてました.良い体験をしました!白髭先輩ありがとうございます.

\end{document}
