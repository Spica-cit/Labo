\documentclass[a4paper,12pt]{ltjsarticle}

% パッケージ
\usepackage{graphicx}
\usepackage{amsmath, amssymb}
\usepackage{xcolor}
\usepackage{geometry}
\usepackage{enumitem}
\usepackage{hyperref}
\usepackage{luatexja-fontspec}

% フォント設定
\setmainjfont{IPAexMincho}
\setmainfont{Times New Roman}
\setsansjfont{IPAexMincho}
\setsansfont{Times New Roman}

% レイアウト
\geometry{margin=25mm}
\setlist[itemize]{label=--, leftmargin=2em}
\setlist[enumerate]{leftmargin=2em}

%====================
% 強調用コマンド
%====================
\newcommand{\important}[1]{\textcolor{blue}{\textbf{#1}}}

\begin{document}

%====================
% タイトルブロック
%====================
\begin{center}
  {\huge 進捗報告(ver.35)}\\[1em]
  {\large \today}\\
  {\large 千葉工業大学大学院 学習工学研究室}\\
  {\large 内山 裕太}
\end{center}

\vspace{2em}

%====================
\section{進捗概要}
研究タイトル?\\
「定式化構造を部品として捉えた方程式文章題の問題解決過程支援」\\

部品の構成は?\\
\begin{itemize}
  \item 数量役割
  \begin{itemize}
    \item 初期量:$A$
    \item 変化量:$B$
    \item 最終量:$x$\\
  \end{itemize}
  \item 関係
  \begin{itemize}
    \item $x=A+B$\\
  \end{itemize}
\end{itemize}

上記のような定式化構造を「部品」として扱う?
\clearpage

$問題例①$\\
「太郎はりんごを3個持っている.花子から2個もらった場合,太郎はりんごを何個持っているか?」\\

\begin{itemize}
  \item 表層要素
  \begin{itemize}
    \item 太郎
    \item 花子
    \item りんご
    \item 3個
    \item 2個\\
  \end{itemize}
  \item 数量関係
  \begin{itemize}
    \item 初期量:$3個$
    \item 変化量:$+2個$
    \item 最終量:$?$\\
  \end{itemize}
  \item 定式化構造
  \begin{itemize}
    \item $?=3+2$\\
  \end{itemize}
\end{itemize}
\clearpage

$問題例②$\\
「A君は500円持っている.お小遣いとして300円もらった.A君の所持金はいくらになったか?」\\

\begin{itemize}
  \item 表層要素
  \begin{itemize}
    \item A君
    \item 500円
    \item 300円
    \item お小遣い\\
  \end{itemize}
  \item 数量関係
  \begin{itemize}
    \item 初期量:$500円$
    \item 変化量:$+300円$
    \item 最終量:$?$\\
  \end{itemize}
  \item 定式化構造
  \begin{itemize}
    \item $?=500+300$\\
  \end{itemize}
\end{itemize}
\clearpage

$問題例③$\\
「家から駅までの道のりは800mである.出発してから300m歩いた地点での残りは何mか?」\\

\begin{itemize}
  \item 表層要素
  \begin{itemize}
    \item 家
    \item 駅
    \item 800m
    \item 300m\\
  \end{itemize}
  \item 数量関係
  \begin{itemize}
    \item 初期量:$300m$
    \item 変化量:$?$
    \item 最終量:$800m$\\
  \end{itemize}
  \item 定式化構造
  \begin{itemize}
    \item $300+?=800$\\
  \end{itemize}
\end{itemize}

\important{変数の位置が異なる(移項とかがあるけど,文脈として考えた場合?)}
\clearpage
全部同じ問題として扱ってよいのでは?(加法だし)\\

\important{自分の研究としてはよくない\\}

\begin{itemize}
  \item 問題①,②
  \begin{itemize}
    \item $?=α+β$\\
  \end{itemize}
  \item 問題③
  \begin{itemize}
    \item $α+?=β$\\
  \end{itemize}
\end{itemize}

\important{式変形したら同じだが,定式化構造としては別物なのでは?\\}

重要な点\\
\begin{itemize}
  \item 「式が同じ」として片づけることもできる
  \item 数量役割と関係の配置が同じ
  \begin{itemize}
    \item \important{定式化した構造が類似している(問題として似ている)\\}
  \end{itemize}
\end{itemize}

学習者自身は何を根拠に「同じ」を判断する?\\

教師「これは前と同じタイプの問題だよ!」\\
学習者「なるほど(わかった気になっている)」\\

but...\\

\begin{itemize}
  \item どこが異なるのか?
  \item どこが同じなのか?\\
\end{itemize}

上記が明示されないと,わからないのでは?

\clearpage
どんなシステムが必要なのだろう?(わかってない)\\

必要そうな機能
\begin{itemize}
  \item 表層構造の違いを示す
  \begin{itemize}
    \item 問題文だけだと全く別の問題に見える
    \item 「解けるのか」ではなく,「構造的な違い」を比較する必要あり\\
  \end{itemize}
  \item 数量役割を明示的に操作
  \begin{itemize}
    \item 各数値に対して役割を割り当てる(初期量,変化量など)\\
  \end{itemize}
  \item 定式化構造を「部品」として可視化する
  \begin{itemize}
    \item 定式化構造自体に名前などを付け,「前にやったことあるかも...」から再利用を促したい(できるのか?)\\
  \end{itemize}
  \item 判定(正解かどうかではなく,理由の重視)
  \begin{itemize}
    \item 部品を選んだ理由は?
    \item 対応している要素は?\\
  \end{itemize}
\end{itemize}
\clearpage

%====================
% \section{前回からの進捗}
% \begin{enumerate}
%   \item 
% \end{enumerate}

%====================
\section{得られた結果・考察}
\begin{itemize}
  \item 部品の抽象化手法において重要な内容
  \begin{itemize}
    \item 表層構造が異なると全く別の問題に見えることがある
    \item 定式化構造が類似するなら理解できるのでは?
  \end{itemize}
\end{itemize}

%====================
\section{問題点・課題}
\begin{itemize}
  \item 具体的なシステム構成が見えていない
  \begin{itemize}
    \item 一旦,学習のフローを考える必要がある
    \item 立式の過程でどのような計画立てで行なっているのか分析(この辺研究している資料とかありそう?)
  \end{itemize}
\end{itemize}

%====================
\section{今後の予定}
\begin{enumerate}
  \item 具体的なシステム案の作成
  \item システムの機能要件の検討
\end{enumerate}
\clearpage

%====================
\section*{雑談}
\begin{itemize}
  \item 朝起きれるが,気分的な問題で目覚めが悪い(悪い夢を見ることが多め)\\
  \item ストレスを感じている自覚はないが,体はピークなのか?わからない
  \begin{itemize}
    \item 研究はモチベが上がっている
    \item 私生活も嫌なことはない\\\\
    となると,職場環境?あまり意識してないが,体力的に悲鳴?他の人に比べてがんばれている自覚はあまりないが...\\
  \end{itemize}
  \item 1月末に卒塾する生徒(1年半担当)から手紙をもらった(泣きましたが,内容が可愛かった)\\
  \item 自分は何を楽しみに生きているのか(ゲーム最近やってないし)
\end{itemize}

%====================
\section*{備考・メモ}
\begin{itemize}
  \item 2/17(火)_研修?いないかもしれない(遅れてこれる可能性(微))
  \item 3/3(火)_合格祝賀会(受験生の合否後にパーティー)\\
  \item to 前田先輩\\
  統計サイトの修正まだしてません(一応)\\
\end{itemize}

今後の研究メモ
\begin{itemize}
  \item MIPSをもう一度読んでみる
  \item 補助問題の定式化との関係性は?(いらないと思う)
  \item 全国数学教育学会(良さそう)
  \item 院ゼミのスレッド読め\\
  \item 数学の研究(証明問題の部品化(システム作っちゃえ)
  \item メンタリング活動の研究(文献調べろ)
\end{itemize}

\end{document}
