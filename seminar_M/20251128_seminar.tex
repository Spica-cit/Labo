\documentclass[a4paper,12pt]{ltjsarticle}
\usepackage{graphicx}
\usepackage{amsmath, amssymb}
\usepackage{geometry}
\usepackage{enumitem}
\setlist[itemize]{label=-}
\usepackage{luatexja-fontspec}
\setmainjfont{IPAexMincho}
\setmainfont{Times New Roman}
\setsansjfont{IPAexMincho}
\setsansfont{Times New Roman}
\geometry{margin=25mm}
\usepackage{hyperref}
\usepackage{color}

\begin{document}

\centerline{\huge 進捗報告 ver.30}

\rightline{\large \today}
\rightline{\large 千葉工業大学大学院}
\rightline{\large 学習工学研究室}
\rightline{\large 内山裕太}

\section*{今回の進捗}
\begin{itemize}
  \item モチベーションが\color{red}激低下\color{black}
  \begin{itemize}
    \item 研究に対する熱量(何がしたいのかわからない)
    \item 就職も含め「何を求められているのか」がわからない
    \begin{itemize}
      \item こちらについては考えを整理してわかってきた
    \end{itemize}
    \item 研究室で活動するのは全く嫌ではないが,やることが見えなくて「行く意味...」ってなってる時があるので,モチベが問題?\\
  \end{itemize}
  \item 高野,勝井の年越しまでの計画立て
  \begin{itemize}
    \item 卒論のメンター提出を追加(5回)
    \item メンターリハ日程組み(12/1)
  \end{itemize}
\end{itemize}

\clearpage

\section*{今後の計画}
\begin{itemize}
  \item 修士研究の計画を立てる
  \begin{itemize}
    \item 実際に支援したいのはどんな場面?
    \item 研究テーマは?
    \item 数学を扱うとして何が一番やりたい?
    \item どんなシステムがあったらいいのだろう?\\
  \end{itemize}
  \item 論文拝読(この辺も何がやりたいか次第だが,ほぼ役立つだろうくらいの感覚)
  \begin{itemize}
    \item モンサクン
    \item 補助問題の定式化
    \item MIPS
  \end{itemize}
\end{itemize}

\section*{問題点・課題}
\begin{itemize}
  \item 研究室での活動がだるく感じる時が増えた
  \begin{itemize}
    \item 問題なのは「研究が面倒だから」「家でだらけたいから」などが問題ではないところ
    \item 研究は答えがないからこそ,自分の今抱えている問題は非常に難しい対処が必要
    \item \color{red}モチベを上げるために,早急に「自分の研究」を深掘りする必要あり
  \end{itemize}
\end{itemize}

\clearpage

\section*{スケジュール}
\noindent \includegraphics[width=120mm]{figs/1.png}\\
\includegraphics[width=120mm]{figs/2.png}\\
\includegraphics[width=120mm]{figs/3.png}

\clearpage

\section*{雑談}
システム作りました

\href{https://spica-cit.github.io/Results_analysis/}{遊びシステム}

\section*{メモ}
\subsection*{TA時の先生との相談メモ}
普段使っているテキストで教えるとしたら何がわかりづらい?

どんなシステムがあったらいいのだろう?それをとりあえず作ってみるのは?

院ゼミで作ったものに意見をもらう

学習者の知ってる知識とか,他の分野/科目の知識と結びつけるように考えるとか

食塩水問題も少し変わっただけでわからない人とかいる

内山の脳を移植したシステム(ネタ)

\end{document}
