\documentclass[a4paper,12pt]{ltjsarticle}

% パッケージ
\usepackage{graphicx}
\usepackage{amsmath, amssymb}
\usepackage{xcolor}
\usepackage{geometry}
\usepackage{enumitem}
\usepackage{hyperref}
\usepackage{luatexja-fontspec}

% フォント設定
\setmainjfont{IPAexMincho}
\setmainfont{Times New Roman}
\setsansjfont{IPAexMincho}
\setsansfont{Times New Roman}

% レイアウト
\geometry{margin=25mm}
\setlist[itemize]{label=--, leftmargin=2em}
\setlist[enumerate]{leftmargin=2em}

%====================
% 強調用コマンド
%====================
\newcommand{\important}[1]{\textcolor{blue}{\textbf{#1}}}

\begin{document}

%====================
% タイトルブロック
%====================
\begin{center}
  {\huge 進捗報告(ver.36)}\\[1em]
  {\large \today}\\
  {\large 千葉工業大学大学院 学習工学研究室}\\
  {\large 内山 裕太}
\end{center}

\vspace{2em}

%====================
\section{進捗概要}
\begin{itemize}
  \item 研究進捗はほとんどなし
  \item 病み期
\end{itemize}
\newpage
%====================
\section{前回からの進捗}
色々とご迷惑をおかけしました...\\
自分でもわからない状態になってました...\\

今回の原因は何だったのか?\\
\important{コミュニケーションの不足(情報共有不足も含む)\\}

研究室,バイト,その他の環境でコミュを取る場面がかなり減った気がする\\

\subsection*{研究室,バイト}
白髭先輩が卒業したのもあるが,あまり研究室で会話を交わすことがなかった(おそらく修士になってから研究活動が嫌だった理由の大半はこれ)\\
白髭先輩との定期mtgを当分やっていなかったのと,自分がこの状況が嫌なことが原因で早くバイトに行っていた\\

+\\

バイト先でも忙しいことが多く,自分なりにできることをやっているけど褒められないことがあって嫌になってた?(ベテランになるほど褒められなくはなりますよね.わかります)\\

つまりは,自分は「寂しがり」ってことだと思います.かまってちゃんなんだと思います.\\

うーん...でも,これで気付けたことがありました\\

\begin{itemize}
  \item 勝井君とも毎日簡単なコミュ取った方が良いのでは?\\
  勝井君が期日に間に合わなかったとき,\important{毎日簡単にでも確認してたら回避できたのでは?}と思いました.定期mtgはその辺も管理できている人の技なんだなと(白髭先輩すごいです...)
\end{itemize}

とりあえず,勝井君が望まない場合は別として,毎日簡単なコミュを取ることを目指してやってみよう!\\

追加して,自分は前から白髭先輩以外に頼ったりしてなかったので,そこも気を付けようと思います(なんか頼りづらかったです.すいません)

\subsection*{バイト}
これも研究室の自分と同じでは?ってなりました.\\

後輩指導する中で,自分が「人と話さないと寂しい」と思うのであれば,後輩自身もコミュ取れなくて困ってることあるのでは?と.つまりは,1人でいる先生に積極的に話をしよう!って思います(1人が好きって人は基本残らないような職場なので).そして,この動き方と考え方が今後の自分を強くすると思います.\\

\subsection*{その他}
あと,このような状況になった要因がもう1つ考えられて...\\

妹が今年度から看護学校通うようになって,めっちゃ忙しくなりました.結果として,家族で何かする時間が急激に減った印象があります.その中で,自分は日曜日くらいしかフルで休める日がなく,\important{自分の時間は自分で使いたい}って思ってました.\\

昨夜,同様の話を母親と話したら,とても共感してくれました.4月からは妹が寮生活になる予定なので,もっと対策が必要になってくるかもしれません(父親は単身赴任だし).うーん,難しいです...僕の配属先が大阪とかになったらどうなってしまうんでしょうか...(ないと思うけど)\\

何か良い案ありますか?
\newpage
%====================
% \section{得られた結果・考察}
% \begin{itemize}
%   \item 
% \end{itemize}

%====================
% \section{問題点・課題}
% \begin{itemize}
%   \item 
% \end{itemize}

%====================
\section{今後の予定}
今後は研究とそれ以外の時間にメリハリを付けます.\\

9:00くらい~17:00くらい(研究室)\\
退出後~22:30くらい(バイト)\\

日曜日は家族と過ごす(何もなければその時次第)\\

あと,研究室にいるときに勝井君と何かしら話します.研究進捗でも,雑談でもなんでも.

%====================
\section*{雑談}
\begin{itemize}
  \item マインクラフト買いました!
  \item バイオハザード新作まで1カ月切った!
  \item ChatGPT-Plusに課金
\end{itemize}

%====================
\section*{備考・メモ}
\begin{itemize}
  \item 
\end{itemize}

\end{document}
